% CS 100 style
% Typical usage (all UPPERCASE items are optional):
%       \input 111pre
%       \begin{document}
%       \MYTITLE{Title of document, e.g., Lab 1\\Due ...}
%       \MYHEADERS{short title}{other running head, e.g., due date}
%       \PURPOSE{Description of purpose}
%       \SUMMARY{Very short overview of assignment}
%       \DETAILS{Detailed description}
%         \SUBHEAD{if needed} ...
%         \SUBHEAD{if needed} ...
%          ...
%       \HANDIN{What to hand in and how}
%       \begin{checklist}
%       \item ...
%       \end{checklist}
% There is no need to include a "\documentstyle."
% However, there should be an "\end{document}."
%
%===========================================================
\documentclass[11pt,twoside,titlepage]{article}
%%NEED TO ADD epsf!!
\usepackage{threeparttop}
\usepackage{graphicx}
\usepackage{latexsym}
\usepackage{color}
\usepackage{listings}
\usepackage{fancyvrb}
%\usepackage{pgf,pgfarrows,pgfnodes,pgfautomata,pgfheaps,pgfshade}
\usepackage{tikz}
\usepackage[normalem]{ulem}
\tikzset{
    %Define standard arrow tip
%    >=stealth',
    %Define style for boxes
    oval/.style={
           rectangle,
           rounded corners,
           draw=black, very thick,
           text width=6.5em,
           minimum height=2em,
           text centered},
    % Define arrow style
    arr/.style={
           ->,
           thick,
           shorten <=2pt,
           shorten >=2pt,}
}
\usepackage[noend]{algorithmic}
\usepackage[noend]{algorithm}
\newcommand{\bfor}{{\bf for\ }}
\newcommand{\bthen}{{\bf then\ }}
\newcommand{\bwhile}{{\bf while\ }}
\newcommand{\btrue}{{\bf true\ }}
\newcommand{\bfalse}{{\bf false\ }}
\newcommand{\bto}{{\bf to\ }}
\newcommand{\bdo}{{\bf do\ }}
\newcommand{\bif}{{\bf if\ }}
\newcommand{\belse}{{\bf else\ }}
\newcommand{\band}{{\bf and\ }}
\newcommand{\breturn}{{\bf return\ }}
\newcommand{\mod}{{\rm mod}}
\renewcommand{\algorithmiccomment}[1]{$\rhd$ #1}
\newenvironment{checklist}{\par\noindent\hspace{-.25in}{\bf Checklist:}\renewcommand{\labelitemi}{$\Box$}%
\begin{itemize}}{\end{itemize}}
\pagestyle{threepartheadings}
\usepackage{url}
\usepackage{wrapfig}
\usepackage{hyperref}
%=========================
% One-inch margins everywhere
%=========================
\setlength{\topmargin}{0in}
\setlength{\textheight}{8.5in}
\setlength{\oddsidemargin}{0in}
\setlength{\evensidemargin}{0in}
\setlength{\textwidth}{6.5in}
%===============================
%===============================
% Macro for document title:
%===============================
\newcommand{\MYTITLE}[1]%
   {\begin{center}
     \begin{center}
     \bf
     CMPSC 100 \\ Computational Expression \\
     Fall 2019\\
     Janyl Jumadinova\\
     \medskip
     \end{center}
     \bf
     #1
     \end{center}
}
%================================
% Macro for headings:
%================================
\newcommand{\MYHEADERS}[2]%
   {\lhead{#1}
    \rhead{#2}
    \immediate\write16{}
    \immediate\write16{DATE OF HANDOUT?}
    \read16 to \dateofhandout
    \lfoot{\sc Handed out on \dateofhandout}
    \immediate\write16{}
    \immediate\write16{HANDOUT NUMBER?}
    \read16 to\handoutnum
    \rfoot{Handout \handoutnum}
   }

%================================
% Macro for bold italic:
%================================
\newcommand{\bit}[1]{{\textit{\textbf{#1}}}}

%=========================
% Non-zero paragraph skips.
%=========================
\setlength{\parskip}{1ex}

%=========================
% Create various environments:
%=========================
\newcommand{\PURPOSE}{\par\noindent\hspace{-.25in}{\bf Purpose:\ }}
\newcommand{\SUMMARY}{\par\noindent\hspace{-.25in}{\bf Summary:\ }}
\newcommand{\DETAILS}{\par\noindent\hspace{-.25in}{\bf Details:\ }}
\newcommand{\HANDIN}{\par\noindent\hspace{-.25in}{\bf Hand in:\ }}
\newcommand{\SUBHEAD}[1]{\bigskip\par\noindent\hspace{-.1in}{\sc #1}\\}
%\newenvironment{CHECKLIST}{\begin{itemize}}{\end{itemize}}

\begin{document}
\MYTITLE{Reading Supplement, 18 January 2019 \\
Tips on Using Linux and the Command Line Interface}

\vspace{-0.2in}
\subsection*{Directories, Folders, CLIs, GUIs}
\vspace{-0.1in}
A ``folder'' is the same thing as a ``directory''---a place to save
files. 

\noindent You can navigate
between folders in one of two ways---by typing commands through the
terminal window (this is sometimes called a ``command line
interface'') or by the more familiar method of clicking and double-clicking
on images of folders and icons representing files (this is an example of
a ``GUI,'' or ``graphical user interface''). 

\noindent In this class we concentrate primarily on the command-line interface, since
this material is valuable to people who go on to become 
``power users''---professional
developers, systems administrators, etc., who want to work as
efficiently as possible. You are free to use the GUI
for your own programming, but instructions will almost always
be given in command line form.

\vspace{-0.2in}
\subsubsection*{Home Directory vs. Desktop}
\vspace{-0.1in}
You have a ``home
directory'' that has the same name as your account, e.g., Jane Smith has
a home directory named ``{\tt smithj}.'' Within this directory is
a subdirectory named ``{\tt Desktop}.'' When you see a file icon or folder
on your screen, that is actually saved in your {\tt Desktop}
subdirectory.

\noindent When you right-click on the screen and choose ``Open in Terminal,'' commands
typed in your
terminal window will be relative to the {\tt Desktop} directory since that's 
where you clicked to open the window. \textbf{\textit{The desktop is 
{\em not} your home directory}}---it is one 
level below it.

\noindent It is okay to store things on the {\tt Desktop}. Just be aware that certain
commands described in Table \ref{linux}, such as the command ``{\tt cd},'' 
will automatically return you to your {\em home directory}
and {\em not} to the {\tt Desktop}.

\vspace{-0.2in}
\subsection*{Using the Terminal Window and Keyboard}
\vspace{-0.1in}
If you would like the terminal window application to appear in your list
of applications on the left (the application ``launcher''), do the following:
\begin{itemize}
\item
Right-click on the screen and select the ``Open in Terminal'' command.
The icon for the terminal application should appear on the left menu---it
looks like this: \includegraphics[width=.4in]{images/terminal}. 
\textbf{\textit{You are not in your home directory---you are in the 
{\tt Desktop} folder!}} To get to the home folder, type the command ``{\tt cd}''.
\item
Right-click on the terminal icon on the left and select ``Add to Favorites''. This should permanently add the terminal window to the launcher.
\end{itemize}

{\em Remember---the next time you launch the Terminal application from
this new icon, it will open in your home directory, not the desktop.}

\vspace{-0.2in}
\subsubsection*{Prompts}
\vspace{-0.1in}
Terminal window commands are typed at 
the {\em prompt}. The prompt is usually some 
combination of your username, your computer (e.g., ``{\tt aldenv100}'')
and the current folder or directory. For instance, in the text below, 
the prompt ``\verb#smithj@aldenv52:~$#'' means that user {\tt smithj},
working at machine {\tt aldenv52}, is in the home directory.
%, denoted by the tilde ``\verb$~$''. 
The command ``{\tt mkdir cs100}'' creates a new directory called ``cs100s2019'', ``{\tt cd cs100s2019}'' changes to directory
{\tt cs100s2019}, and the new prompt indicates this. Now typing the command ``{\tt mkdir practical1}'' creates a new subdirectory called ``practical1'' inside ``cs100s2019'' directory, 
``{\tt cd practical1}'' changes to the {\tt practical1} subdirectory, and the prompt now
shows this as well:
\vspace{-0.1in}
\begin{Verbatim}[commandchars=\\\{\}]
     smithj@aldenv52:~$ mkdir cs100s2019           \textcolor{red}{\rm\em [in home directory, make a new directory]}
     smithj@aldenv52:~$ cd cs100s2019              \textcolor{red}{\rm\em [move to cs100s2019 directory]}
     smithj@aldenv52:~/cs100s2019~$ mkdir practical1 \textcolor{red}{\rm\em [in the cs100s2019, make a new subdirectory]}
     smithj@aldenv52:~/cs100s2019$ cd practical1   \textcolor{red}{\rm\em [move to practical1 subdirectory]}
     smithj@aldenv52:~/cs100s2019/practical1$      \textcolor{red}{\rm\em [now in practical1 sub-subdirectory]}
      ... etc. ...
\end{Verbatim}
\vspace{-0.1in}
If you ever see a prompt like ``\verb$smithj@aldenv100:~/Desktop$'' then you
know that you are in the {\tt Desktop} subdirectory of your home directory.
%(NOTE: there are ways to change the prompt, so don't be surprised if some day
%you look at somebody else's screen and see a prompt like 
%``{\tt What can I do for you?}.'')

\vspace{-0.2in}
\subsubsection*{Arrow Keys}
\vspace{-0.1in}
The ``arrow keys'' at the lower right of your keyboard enable you to
move back through the ``history'' of commands you have typed. For instance,
it is often the case, when developing a program, that you will
have to repeatedly type certain commands
as you work on your program. There is no need to keep retyping these
commands! Using the up-arrow key (`` $\uparrow$ '') will let you rapidly
move back to an earlier command so all you need to do is press the 
``Enter'' or ``Return'' key. You can also use the left-arrow and 
right-arrow keys (``$\leftarrow$'' and ``$\rightarrow$'') to move back
and forth in a typed command to correct typing errors or to use a
modified version of an earlier command. 
%(Power users don't like to keep
%retyping long commands---they'd rather use a few simple keystrokes.)

\vspace{-0.2in}
\subsubsection*{Special Keys}
\vspace{-0.1in}
The ``Ctrl'' key (bottom row, below the ``shift'' key) is used in
conjunction with other keys for special commands. For
instance, the key combination Ctrl-C will ``kill'' a process that is
running in the terminal window. (Why would you ever want to do that?
We will see that if you write a program that goes into an ``infinite
loop,'' you must terminate it with a Ctrl-C command.)

\noindent The key combination Ctrl-Z will {\em suspend} a process that is
running in the terminal window; it can later be {\em resumed} in a
number of ways. This is occasionally useful as we will see later.

\noindent If you press the key labeled ``PrtScn'', the computer 
will take a ``snapshot'' of the screen and save it in 
folder named {\tt Pictures} (another subdirectory of your home directory).
If you hold down the ``Shift'' key while pressing the ``PrtScn'' button,
a plus-shaped cursor will appear and you can use it and the left mouse 
button to select a rectangular area of the screen. When you release the
mouse button, you will have a snapshot of the selected area.

\vspace{-0.1in}
\subsubsection*{Linux Commands}
\vspace{-0.1in}
Table \ref{linux} shows some common Linux commands. You'll note 
that many of them are just two letters (``{\tt rm}'' for ``remove,''
``{\tt cp}'' for ``copy,'' etc.). 
%Power users are lazy---they prefer short
%commands!

\begin{table}[htbp]
%\begin{center}
\centering
\begin{tabular}{l|p{1.5in}|p{2.5in}}
\multicolumn{1}{c}{\bf Command} & \multicolumn{1}{c}{\bf Meaning} &
\multicolumn{1}{c}{\bf Example}\\\hline
\rule{0em}{2.5em}\tt ls & List files in current directory & 
\begin{minipage}{2.5in}
\begin{Verbatim}[commandchars=\\\{\}]
jjumadinova@aldenv113:~/cs100/lab1\$ \textcolor{red}{ls}
Lab1.class Lab1.java
\end{Verbatim}
\end{minipage}\\\hline
\rule{0em}{1.5em}\tt mkdir {\rm \em name} & Make a new directory called {\em name}
in the current directory &
\begin{minipage}{2.5in}
\begin{Verbatim}[commandchars=\\\{\}]
jjumadinova@aldenv113:~/cs100\$ \textcolor{red}{mkdir lab2}
\end{Verbatim}
\end{minipage}\\\hline
\rule{0em}{1.5em}\tt cd {\rm \em name} & Change to directory {\em name}
in the current directory &
\begin{minipage}{2.5in}
\begin{Verbatim}[commandchars=\\\{\}]
jjumadinova@aldenv113:~/cs100\$ \textcolor{red}{cd lab2}
jjumadinova@aldenv52 ~/cs100/lab2$
\end{Verbatim}
\end{minipage}\\\hline
\rule{0em}{1.5em}\tt cd & Change to home directory &
\begin{minipage}{2.5in}
\begin{Verbatim}[commandchars=\\\{\}]
jjumadinova@aldenv113:~/cs100\$ \textcolor{red}{cd}
jjumadinova@aldenv113:~\$
\end{Verbatim}
\end{minipage}\\\hline
\rule{0em}{2.5em}\tt cd \verb$..$& Change to directory one level up&
\begin{minipage}{2.5in}
\begin{Verbatim}[commandchars=\\\{\}]
jjumadinova@aldenv113:~/cs100/lab1\$ \textcolor{red}{cd ..}
jjumadinova@aldenv113:~/cs100\$
\end{Verbatim}
\end{minipage}\\\hline
\rule{0em}{4.5em}\tt cp {\rm \em name1 name2} & Copy file {\em name1} to {\em name2}&
\begin{minipage}{2.5in}
\begin{Verbatim}[commandchars=\\\{\}]
jjumadinova@aldenv113:~\$ \textcolor{red}{ls}
Fun.class Fun.java
jjumadinova@aldenv113:~\$ \textcolor{red}{cp Fun.java Lab2.java}
jjumadinova@aldenv113:~\$ \textcolor{red}{ls}
Fun.class Fun.java Lab2.java
\end{Verbatim}
\end{minipage}\\\hline
\rule{0em}{4.5em}\tt mv {\rm \em name1 name2} & Rename file {\em name1} as {\em name2}&
\begin{minipage}{2.5in}
\begin{Verbatim}[commandchars=\\\{\}]
jjumadinova@aldenv113:~\$ \textcolor{red}{ls}
Fun.class Fun.java
jjumadinova@aldenv113:~\$ \textcolor{red}{mv Fun.java Lab2.java}
jjumadinova@aldenv113:~\$ \textcolor{red}{ls}
Fun.class Lab2.java
\end{Verbatim}
\end{minipage}\\\hline
\rule{0em}{4.5em}\tt rm {\rm \em name} & Remove file {\em name}&
\begin{minipage}{2.5in}
\begin{Verbatim}[commandchars=\\\{\}]
jjumadinova@aldenv113:~\$ \textcolor{red}{ls}
Fun.class Lab2.java
jjumadinova@aldenv113:~\$ \textcolor{red}{rm Fun.class}
jjumadinova@aldenv113:~\$ \textcolor{red}{ls}
Lab2.java
\end{Verbatim}
\end{minipage}\\
\hline
\rule{0em}{3.5em}\tt pwd & Shows which directory you are in &
\begin{minipage}{2.5in}
\begin{Verbatim}[commandchars=\\\{\}]
jjumadinova@aldenv113:~\$ \textcolor{red}{pwd} \\
/home/j/jjumadinova/Desktop
\end{Verbatim}
\end{minipage}\\
\hline
\rule{0em}{1.5em}\tt exit  & Close the terminal  &\\
\hline
\end{tabular}
%\end{center}
\caption{Some Common Linux Commands}
\label{linux}
\end{table}

\end{document}
