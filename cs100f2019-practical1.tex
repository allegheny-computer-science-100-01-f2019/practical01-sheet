\input{100pre}
\newcommand{\command}[1]{``\lstinline{#1}''}
\newcommand{\program}[1]{\lstinline{#1}}
%\newcommand{\url}[1]{\lstinline{#1}}
\newcommand{\channel}[1]{\lstinline{#1}}
\newcommand{\option}[1]{``{#1}''}
\newcommand{\step}[1]{``{#1}''}


\long\def\omitit#1{}

\begin{document}
\MYTITLE{Practical 1 \\ 30 August 2019 \\ Due  by midnight of the day of your practical \\ ``Checkmark'' grade}

%\vspace*{-.2in}
\subsection*{Summary}
%\vspace*{-.05in}
To learn how to set up a container-based platform, called Docker, for use in all class activities, labs and practicals. To configure Github and Github Classroom and a text editor called, Atom, for use in the class. To continue to practice using Slack to support communication with the technical leaders and the course instructor.

\vspace*{-.2in}
\subsection*{Configuring Git and GitHub}
\vspace*{-.1in}

During the subsequent practical and  laboratory assignments, we will securely communicate
with the GitHub servers that will host all of the project templates and your submitted deliverables. In this assignment,
you will perform all of the steps to download Git and configure your account on GitHub, so that you are ready to start your first lab assignment using
GitHub Classroom next week.  You can
also learn more about GitHub Classroom by visiting \url{https://classroom.github.com/}. As you will be required to use
Git, an industry standard tool, in all of the  laboratory and remaining practical assignments and during the class
sessions, you should keep a record of all of the steps that you complete and the challenges that you face. You may see
the course instructor or one of the teaching assistants if you are not able to complete a certain step or if you are not
sure how to proceed.

\begin{enumerate}

  \item If you do not already have a GitHub account, then please go to the GitHub website (\url{https://github.com/}) and create one, making sure
    that you use your \command{allegheny.edu} email address so that you can join GitHub as a student at an accredited
    educational institution. You are also encouraged to sign up for GitHub's ``Student Developer Pack'' at
    \url{https://education.github.com/pack}, qualifying you to receive free software development tools. Additionally,
    please add a description of yourself and an appropriate professional photograph to your GitHub profile. Unless your
    username is taken, you should also pick your GitHub username to be the same as Allegheny's Google-based email
    account. 
  
  \item Install \href{https://git-scm.com/downloads}{git} for the operating system of your laptop. We recommend you complete the following steps.
  
  \textbf{For Windows:} 
  \begin{itemize}
  	\item Open a terminal (command prompt) window in administrator mode. To do this, search for "cmd" in the start menu, then right-click and select ``Run as Administrator". 
  	\item Then, copy and paste the following command into that terminal, and hit enter.   
  	\begin{verbatim}
  	$ @"%SystemRoot%\System32\WindowsPowerShell\v1.0\powershell.exe" -NoProfile -InputFormat None -ExecutionPolicy Bypass -Command "iex ((New-Object System.Net.WebClient).DownloadString('https://chocolatey.org/install.ps1'))" && SET "PATH=%PATH%;%ALLUSERSPROFILE%\chocolatey\bin"
  	\end{verbatim}
	\item To test that this step was completed successfully, open a new terminal, and type the command \command{choco --version}. If 0.10.11 is printed (or some other higher version), you have completed this step. If not, ask for the instructor or a technical leader for assistance.
	\item Run the following commands in a terminal window in administrator mode. \\
	{\tt choco install git -y --params "/GitAndUnixToolsOnPath} \\
	{\tt /WindowsTerminal /NoShellIntegration"}
	\item To test that this step was completed successfully, open a new terminal and type: \\ {\tt git --version}. \\
	If git version followed by some version number is printed (or something similar), you have completed this step.
	\end{itemize}
	
	[TODO: add Mac and Linux instructions]

  \item If you have never done so before, you must use the \command{ssh-keygen} program to create secure-shell keys that
    you can use to support your communication with GitHub. But, to start, this task requires you to type commands in a terminal. Open the terminal, a window in which you can type commands. which is called 

  \item Now that you have started the terminal, you will now need to type the \\
  \command{ssh-keygen -t rsa -b 4096 -C "your\_email\_used\_to\_create_github_account@allegheny.edu"}
   command in it. Follow
    the prompts to create your keys and save them in the default directory. That is, you should press ``Enter'' after
    you are prompted to \command{Enter file in which to save the key ...  :} and then type your selected passphrase
    whenever you are prompted to do so. Please note that a ``passphrase'' is like a password that you will type when you
    need to prove your identify to GitHub. What files does \command{ssh-keygen} produce? Where does this program store
    these files by default? Do you have any questions about completing this step?

  \item Once you have created your ssh keys, you can raise your hand to invite either a technical leader or the
    course instructor to help you with the next steps as needed. First, you must log into GitHub and look in the right corner for
    an account avatar with a down arrow. Click on this link and then select the ``Settings'' option. Now, scroll down
    until you find the ``SSH and GPG keys'' label on the left, click to create a ``New SSH key'', and then upload your
    ssh key to GitHub. You can copy your SSH key to the clipboard by going to the terminal and typing ``{\tt cat
    \textasciitilde{}/.ssh/id\_rsa.pub}'' command and then highlighting this output. When you are completing this step
    in your terminal window, please make sure that you only highlight the letters and numbers in your key---if you
    highlight any extra symbols or spaces then this step may not work correctly. Then, paste this into the GitHub text
    field in your web browser.

  \item Again, when you are completing these steps, please make sure that you take careful notes about the inputs,
    outputs, and behavior of each command. If there is something that you do not understand, then please ask the course
    instructor or the teaching assistant about it.

\end{enumerate}

\vspace*{-.2in}
\subsection*{Using a Container-based Platform, Docker}
\vspace*{-.1in}

Docker is a platform for developers and system administrators to develop, deploy, and run software applications with containers. [TODO: add more flowery language about Docker]
%Docker is becoming a disruptive technology and revolutionizing the IT industry. 

The instructor in this course will deploy Docker containers with all the necessary software to run all class exercises, lab and practical assignments. Before using Docker, students must first complete the installations relevant to their operating system. 

\begin{enumerate}
	\item As the first step of the Docker set up process, please check and make a note of the version of your operating system. If your operating system meets the requirements outlined below, then proceed to the next step. For older MacOS and Windows Operating Systems, see the next section.
	\item Then, go to \url{https://docs.docker.com/install/} and  from the menu on the left-hand side select and follow the installation tutorial for your operating system (Linux, MacOS, or Windows). Please see additional notes below before proceeding.
	\item Once your installation is complete, please run {\tt > docker run hello-world} command in the terminal window. If you see an output similar to the one below, feel free to high five a technical leader - your installation was successful!

\begin{verbatim}	
docker run hello-world

docker : Unable to find image 'hello-world:latest' locally
...

latest:
Pulling from library/hello-world
ca4f61b1923c:
Pulling fs layer
ca4f61b1923c:
Download complete
ca4f61b1923c:
Pull complete
Digest: sha256:97ce6fa4b6cdc0790cda65fe7290b74cfebd9fa0c9b8c38e979330d547d22ce1
Status: Downloaded newer image for hello-world:latest

Hello from Docker!
This message shows that your installation appears to be working correctly.
...
\end{verbatim}

\end{enumerate}

\begin{itemize}
	\item Docker Desktop for Mac and Windows requires you to create an account to download. To bypass this, use the following download links:
	\begin{itemize}
		\item Windows: \url{ https://download.docker.com/win/stable/Docker%20for%20Windows%20Installer.exe}
		\item Mac: \url{ https://download.docker.com/mac/stable/Docker.dmg}
	\end{itemize}
\end{itemize}
	

\subsubsection*{Older Mac and Windows systems}
Older Mac and Windows operating systems that do not meet the requirements of Docker Desktop for Mac and Docker Desktop for Windows can use Docker Toolbox. You can follow the tutorial on \url{docs.docker.com/toolbox/toolbox_install_windows/}, while keeping the following steps in mind.

\begin{enumerate}
	\item Make sure your operating system version satisfies the requirements for the Docker Desktop.
	\item Check that virtualization is running using Specci. If not enabled, then the BIOS settings must be changed. For example, to change this setting on an HP Windows 10 Home, you should first shut down your computer. Secondly, you turn on your machine, while holding the {\tt esc} button and wait for the BIOS menu to appear. Then, select the {\tt F10} key for the BIOS configuration. Now you should be in BIOS settings, where you can use the arrow keys to navigate to the ``System configuration'' tab and select ``Virtual Technology'' entry. Push {\tt enter} and select ``Enabled.'' Finally, use {\tt F10} key to ``Save and Exit'' and reboot your computer. Please note that the specific keys and operations maybe different on your machine.
	\item During  the toolbox installation, there are several different items to select for installation. Make sure  that the (Oracle) VirtualBox option is checked.
\end{enumerate}

\vspace*{-.2in}
\subsection*{Text Editor for Programming}
\vspace*{-.1in}
In this class we will use a text editor called, {\tt atom}, to write our programs. Please go to \url{https://atom.io}. You should see a Download button specific to your operating system.

\omitit{
\vspace*{-.2in}
\subsection*{Navigating using the Command Line Interface}
\vspace*{-.1in}
A command-line interface allows the user to interact with the computer by typing in commands. Computing professionals prefer to use the command line interface, built into operating systems like Linux, instead of using the graphical user interface. In many situations command line interface tends to be very efficient and effective, for example, it allows you to complete some tasks with a simple one line command instead of using the ``pumping'' motion of the mouse!
\vspace*{-.1in}
\begin{enumerate}
\item Read through the supplemental handout on ``Tips on Using Linux and the Command Line Interface''. Locate the terminal window and open it as explained in the reading handout. 
\item Now you will practice using the commands discussed in the handout. Follow the handout to try the commands discussed  using the terminal window. Then, type each of the commands found in Table 1 of the supplemental handout. Make sure you understand what each command does. You will have to create new files in order to run some commands such as {\tt cp}, {\tt mv}, etc. The most basic method of creating a file is with the {\tt touch} command. This will create an empty file using the name specified: {\tt touch file1} or multiple files as: {\tt touch file1 file2}. 
Remember to execute a command, you should press ``Enter'' after typing a command. Check with your neighbors to see if they are able to open the terminal window, and use commands such as {\tt cd, cp, pwd, ... , ls}, etc.
\item After you finish practicing using the terminal commands delete all of the newly created files and directories from the previous practice step to avoid confusion in the future. 
\item Now, if you have not done so in the previous step, create a directory  called {\tt cs100s2019}  in your home directory, by typing {\tt mkdir cs100s2018} command in your terminal.  This is where all of the work you do in this class should reside. 
\item You can now close the terminal window by typing the {\tt exit} command. 
\end{enumerate}
}


\noindent Since this is your first  assignment and you are still learning how to use the appropriate software,
    don't become frustrated if you make a mistake. Instead, use your mistakes as an opportunity for learning both about
    the necessary technology and the background and expertise of the other students in the class, the teaching
    assistants, and the course instructor. Remember, you can use Slack to talk with the instructor by typing
    \command{@jjumadinova} in a channel.
 

%\vspace*{-.25in}
\subsection*{General Guidelines for Practical Sessions}
%\vspace*{-.05in}
\begin{itemize}
\item {\bf Experiment!} Practical sessions are for learning by doing without the pressure of grades or ``right/wrong''
  answers. So try things!  The best way to learn is by trying things out.

\item {\bf Complete \textbf{\textit{Something}}.} Your grade for this assignment is a ``checkmark'' indicating whether you
  did or did not complete the work.

\item {\bf Practice Key Laboratory Skills.} As you are completing this assignment, practice using the Ubuntu terminal until you can easily use its most important features.  Additionally, ask
  a teaching assistant or a course instructor to teach you some of the advanced features of the terminal, thereby helping you to work more effectively. 

\item {\bf Try to Finish During the Class Session.} Practical exercises are not intended to be the equal of the
  laboratory assignments. If you are simply a slow typist, I've given you until the end of the day, but ideally you
  should complete the assignment by the end of the class period. 

\item {\bf Help One Another!} If your neighbor is struggling and you know what to do, offer your help. Don't ``do the
  work'' for them, but advise them on what to type or how to handle things. If you are stuck on a part of this practical
  session and you could not find any insights in either your textbook or online sources, formulate an intelligent
  question to ask your neighbor, a teaching assistant, or a course instructor. Try to strike the right balance between
  asking for help when you cannot solve a problem and working independently to find a solution.
\end{itemize}

\end{document}
